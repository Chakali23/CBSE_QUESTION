\documentclass{article}
\usepackage{amsmath}
\usepackage{multicol}

\title{\textbf{Linear}}
\date{}

\begin{document}

\maketitle

\begin{enumerate}

	\item \textbf{Assertion (A):} Point P(0,2) is the point of intersection of $y-axis$ with  the line $3x+2y=4$.\\
		\textbf{Reason (R):} The distance of point P(0,2) from $x-axis$ is 2 units.

	\item If the pair of equations $3x-y+8=0$ and $6x-ry+16=0$ represent coincident lines, then the value of \text{'$r$'} is:
		\begin{multicols}{2}
		\begin{enumerate}
			\item $-\frac{1}{2}$
			\item $\frac{1}{2}$
			\item -2
			\item 2
		\end{enumerate}
		\end{multicols}

	\item The of linear equations $2x=5y+6$ and $15y=6x-18$ represents two lines which are:
		\begin{multicols}{2}
			\begin{enumerate}
				\item intersecting
				\item parallel
				\item coincident
				\item either intersecting or parallel
			\end{enumerate}
		\end{multicols}

	\item (a) Find the equations of the diagonals of the parallelogram PQRS whose vertices are P(4,2,-6), Q(5,-3,1), R(12,4,5) and S(11,9,-2). Use these equations to find the point of intersection of diagonals.

		$$\textbf{OR}$$

		(b) A line $l$ passes through point (-1,3,-2) and is perpendicular to both the lines $\frac {x}{1}=\frac{y}{2}=\frac{z}{3}$ and $\frac {x+2}{-3}=\frac{y-1}{2}=\frac{z+1}{5}$. Find the ctor equation of the line $l$. Hence, obtain its distance from origin.

\end{enumerate}
\end{document}
